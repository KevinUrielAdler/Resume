% Curriculum Vitae
% Kevin Uriel Manzano Rios

\documentclass[10pt,a4paper,ragged2e,withhyper]{altacv}

\geometry{left=1.25cm,right=1.25cm,top=1.5cm,bottom=1.5cm,columnsep=1.2cm}
\usepackage{paracol}
\usepackage[default]{lato}

\definecolor{VividPurple}{HTML}{3E0097}
\definecolor{SlateGrey}{HTML}{2E2E2E}
\definecolor{LightGrey}{HTML}{666666}
\definecolor{ColorLinks}{HTML}{0000ff}

\colorlet{heading}{VividPurple}
\colorlet{headingrule}{VividPurple}
\colorlet{accent}{VividPurple}
\colorlet{emphasis}{SlateGrey}
\colorlet{body}{LightGrey}

\renewcommand{\itemmarker}{{\small\textbullet}}
\renewcommand{\ratingmarker}{\faCircle}
\renewcommand{\cvDateMarker}{\faCalendar*[regular]}
\renewcommand{\cvLocationMarker}{\faMapMarker*}

\begin{document}
\name{Kevin Uriel Manzano Rios}
\personalinfo{
  \email{kmanzanor24@gmail.com}
  \phone{+(52) 56 1748 0590}
  % \homepage{https://github.com/KevinUrielAdler}
  \github{KevinUrielAdler}
  \linkedin{KevinUrielManzanoRios}
  \location{México}
}

\makecvheader
\AtBeginEnvironment{itemize}{\small}
\columnratio{0.6}
\begin{paracol}{2}

  \cvsection{Experiencia}

  \cvachievement{\faLaptopCode}{Desarrollador Python y de Procesamiento de
  Lenguaje Natural en Rombo Works}{Lideré el equipo de búsqueda de información para
  el desarrollo de un sistema RAG (Retrieval-Augmented Generation) que alimentaba
  un chatbot avanzado capaz de responder consultas en lenguaje natural acerca del
  Diario Oficial de la Federación de México. Implementé soluciones de recuperación
  inteligente con Python, haciendo uso de bases de datos vectoriales y APIs de
  lenguaje natural, logrando respuestas completas y estructuradas,
  independientemente del periodo o tipo de consulta.}{Ene 2024 -- Oct 2024}
  \divider

  \cvachievement{\faChalkboardTeacher}{Profesor en la Vocacional}{Durante mi
  servicio social diseñé e impartí lecciones de C/C++, algoritmos, y resolución
  de problemas a 60 estudiantes en el Club de Algoritmia. Además,
  enseñé a utilizar el OMIbot a estudiantes interesados en la Olimpiada
  Mexicana de Informática (OMI).}{Ago 2020 -- Ago 2021}

  \cvsection{Proyectos Relevantes}

  \cvachievementalt{\faBook}{Proyecto de Algorithm Learning}
  {Desarrollé {\href{https://github.com/KevinUrielAdler/Allearning}{\color{ColorLinks}
  "Allearning"\color{black}}}, una plataforma para enseñar programación
  competitiva en C++. La aplicación ofrece a los usuarios lecciones y
  ejercicios interactivos. Para complementar la experiencia y hacerla accesible
  desde cualquier dispositivo, creé una
  {\href{https://github.com/KevinUrielAdler/Allearning-Web}{\color{ColorLinks}
  página web\color{black}}} utilizando .NET.}
  \divider

  \cvachievementalt{\faChild}{Asistente Virtual Conversacional "ZAID"}
  {Creé {\href{https://github.com/KevinUrielAdler/AvZ}{\color{ColorLinks}
  ZAID\color{black}}}, un asistente virtual conversacional utilizando Python.
  Implementé su capacidad para interactuar con el usuario en lenguaje natural y
  realizar diversas tareas automatizadas. El sistema fue diseñado para funcionar
  en segundo plano y responder a comandos de voz de forma inteligente.}
  % \divider

  % \cvachievementalt{\faCalculator}{Calculadora de Máximos y Mínimos}
  % {Creé una {\href{https://github.com/KevinUrielAdler/Maximos-y-Minimos}
  % {\color{ColorLinks}herramienta\color{black}}} para calcular máximos y mínimos de
  % funciones escalares de variable vectorial usando SymPy. La herramienta también
  % genera un reporte detallado del procedimiento y los resultados en LaTeX, los
  % cuales se presentan en un archivo PDF.}

  \cvsection{Formación}
  \cvevent{Ingeniería en Inteligencia Artificial}
  {\href{https://es.wikipedia.org/wiki/Escuela_Superior_de_Cómputo}
  {Escuela Superior de Cómputo (\color{ColorLinks}ESCOM, IPN\color{black})}}
  {Ene 2022 -- Previsto Dic 2025}{Av. Juan de Dios Bátiz 46188, CDMX}
  \begin{itemize}
    \item Especialización prevista: Ciencias de la Computación
  \end{itemize}
  \divider
  \cvevent{{\href{https://drive.google.com/file/d/1U6rwow80xCz0LrepnSFyvGuc5V7Hr1Id/view?usp=sharing}{\color{ColorLinks}Carrera Técnica\color{black}}} en Informática}
  {{\href{https://www.cecyt13.ipn.mx}
  {Centro de Estudios Científicos y Tecnológicos N°13 (\color{ColorLinks}IPN\color{black})}}}
  {Ago 2018 -- Ago 2021}{Calz Taxqueña 1620, CDMX}
  \begin{itemize}
    \item {\href{https://drive.google.com/file/d/1osz7QmjWUH6OcZ8AOrqVXEIJf-Pet3Q5/view?usp=sharing}{\color{ColorLinks}Promedio
    general\color{black}}}: 82 de 100
  \end{itemize}
  \divider
  \cvevent{Cursos de Python 3 y Azure}{}{2021}{}
  \begin{itemize}
    \item Completé un curso oficial de Microsoft para certificarme en
    {\href{https://drive.google.com/file/d/1iRQRtXmNAItFIWmfjOP1m_35C4pAPpTP/view?usp=sharing}{\color{ColorLinks}"Microsoft
    Azure Fundamentals AZ 900"\color{black}}} y un curso de Python usando python
    3, SQLite, y Flask.
  \end{itemize}

  \switchcolumn

  \cvsection{Tecnologías}
  \cvtag{C/C++ (5 años)}
  \cvtag{Python (5 años)}\\
  \cvtag{Rust (6 meses)}
  \cvtag{TensorFlow (1 año)}\\
  \cvtag{PyTorch (2 años)}
  \cvtag{C\# (1 año)}\\
  \cvtag{Docker (1 año)}
  \cvtag{Linux (2 años)}\\
  \cvtag{HTML (3 años)}
  \cvtag{MySQL/SQLite (2 años)}\\
  \cvtag{Chroma DB (1 año)}
  \cvtag{Git/GitHub (5 años)}

  \cvsection{Aptitudes}
  \cvtag{Trabajo en Equipo}
  \cvtag{Arquitectura de Software}\\
  \cvtag{Optimización de Código}
  \cvtag{Enfoque Lógico}
  \cvtag{Iniciativa}
  \cvtag{Resolución de Problemas}
  \cvtag{Creatividad}

  \cvsection{Idiomas}
  \begin{tabular}{l | ll}
    \textbf{Español} & Lengua materna\\
    \textbf{Inglés} & \href{https://drive.google.com/file/d/1gBM0AOUTz7h87_4FS4IsUOT8kphNWD5m/view?usp=sharing}{\color{ColorLinks}B2 - Intermedio alto (MCER)\color{black}}\\
    \textbf{Alemán} & A2 - Elemental (MCER)
  \end{tabular}

  \cvsection{Premios y\\ concursos}
  \cvevent{{\href{https://drive.google.com/file/d/1a9Y24AIVQ5aGS7OPvPD1qg4wDesXFmRi/view?usp=sharing}{\color{ColorLinks}Segundo Lugar\color{black}}} (Concurso estatal)}
  {Olimpiada Mexicana de Informática
  ({\href{https://www.olimpiadadeinformatica.org.mx/OMI/OMI/InfoGeneral/Que_es_la_OMI.aspx}{\color{ColorLinks}OMI\color{black}}})}
  {}{}
  \begin{itemize}
    \item Completé el curso impartido por el Centro de Innovación Tecnológica
    Educativa (CITE) sobre programación competitiva en C++.
    \item Participé en la Olimpiada Mexicana de Informática y obtuve la medalla
    de plata a nivel estatal.
  \end{itemize}
  \divider
  \cvevent{Concursos de Robótica}{Robot War, MakeX}{}{}
  \begin{itemize}
    \item Participé en 2 concursos de robótica,
    {\href{https://drive.google.com/file/d/1xETjexZx_X1Bg46y_-m8kKneEbaz48dQ/view?usp=sharing}{\color{ColorLinks}
    guerra de robots\color{black}}} y
    {\href{https://drive.google.com/file/d/1QcXzJbDVQ1dSvtQrdd_j387KvKlfXH8J/view?usp=sharing}{\color{ColorLinks}
    MakeX Robotics Competition\color{black}}}, patrocinados por MISUMI y
    CreativaKids respectivamente.
  \end{itemize}
  \divider
  \cvevent{Competición Internacional Universitaria de Programación}
  {{\href{https://es.wikipedia.org/wiki/Competición_Internacional_Universitaria_de_Programación}{\color{ColorLinks}ICPC\color{black}}}}
  {}{}
  \begin{itemize}
    \item Participé en el ICPC en 2020,
    {\href{https://drive.google.com/file/d/1OwYn_YItOoTw_DI-Dajlrwc_oC57Rv6G/view?usp=sharing}{\color{ColorLinks}
    nuestro equipo\color{black}}} llegó al top 100 en la final.
  \end{itemize}

  \newpage
\end{paracol}
\end{document}
